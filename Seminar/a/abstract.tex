
\begin{titlepage}
\begin{center}
\textbf{\LARGE{Abstract}}\\[1cm]
\end{center}
\normalsize
\par Green transportation such as electric vehicles are emerging as an alternative to the traditional vehicles primarily due to the increasing cost and need of petroleum energy worldwide. These electric vehicles operate by using electric charging and the way to charge an electric car is to use a mobile charger or use a charging infrastructure. Therefore, when a mobile charger is used, a billing system is required through which a user is billed who has charged the electric vehicle. In this paper, I propose a mobile charger billing system that utilizes Blockchain technology. This technology has been applied to achieve more secure online transactions in a peer-to-peer manner. Moreover, it analyzes the requirements of mobile charger for billing and propose a lightweight scheme that can overcome the challenge of data size in existing Blockchain. Current online transaction rely on certain trusted institutions. However, these third party sources can be hacked, manipulated or compromised. There is a need for an electronic payment system for direct transactions between trading partners, without the existence of a trusted third
party. Blockchain is a technique that enables reliable electronic transactions by creating computational evidence of the time sequence of transactions using a peerto-peer distributed timestamp server to solve this problem. They explain electronic cash which is dealt in peer-to-peer network so that direct transactions can be made
between the two parties without trading through a third trusted institution. A Blockchain is essentially a public ledger that is executed and shared between participants. Once the data has been entered into the block, it is difficult to forge or delete the information. If a malicious user attempts to modify or delete a block, it must also modify all previous blocks as well as the block at that point in time.
\end{titlepage}
