\documentclass[10pt]{beamer}

\usetheme{metropolis}
\usepackage{appendixnumberbeamer}

\usepackage{booktabs}
\usepackage[scale=2]{ccicons}

\usepackage{pgfplots}
\usepgfplotslibrary{dateplot}

\usepackage{xspace}
\newcommand{\themename}{\textbf{\textsc{IoT}}\xspace}

\title{Securing IOT devices using Blockchain}
%\subtitle{A modern beamer theme}
\date{\today}
\author{Rafsal Rahim \newline TVE16MCA41}
\institute{Dept. MCA, College of Engineering Trivandrum}
% \titlegraphic{\hfill\includegraphics[height=1.5cm]{logo.pdf}}

\begin{document}

\maketitle

\begin{frame}{Table of contents}
  \setbeamertemplate{section in toc}[sections numbered]
  \tableofcontents[hideallsubsections]
\end{frame}

\section{Introduction}

\begin{frame}[fragile]{What do IOT  mean?}
\metroset{block=fill}
      \begin{block}{Definition}
	The internet of things is a system of interrelated computing devices that are provided with unique identifiers (UIDs) and the ability to transfer data over a network without requiring human-to-human or human-to-computer interaction.
	\end{block}
	IoT architecture can be represented by four building blocks:
	\begin{itemize}
		\item \textsc{Things}
		\item \textsc{Gateways}
		\item \textsc{Network infrastructure}
		\item \textsc{Cloud infrastructure}
	\end{itemize}



 % \begin{verbatim}    \documentclass{beamer}
  %  \usetheme{metropolis}
   % \end{verbatim}

\end{frame}

\begin{frame}{Figures 1}
  \begin{figure}  
	\includegraphics[width=7cm]{iot_struct}    
    \caption{building blocks of IoT}  \end{figure}
\end{frame}

\section{Challenges}

\begin{frame}[fragile]{Challenges to secure IoT deployments}
	\begin{itemize}
		\item \texttt{\textbf{\themename Systems are poorly designed}}
		\item \texttt{\textbf{complex and
sometimes conflicting configurations}}
		\item \texttt{\textbf{Limited guidance for life cycle maintenance and management of IoT devices}}
		\item \texttt{\textbf{There is a lack of standards for authentication and authorization of IoT edge devices.}}
				\item \texttt{\textbf{denial-of-sleep attacks}}
		\item \texttt{\textbf{denial-of-service attacks (DoS) attacks}}
	\end{itemize}
\end{frame}


\begin{frame}[fragile]{Problem with current centralized model}
	\begin{itemize}
		\item \texttt{\textbf{Current IoT ecosystems rely on centralized, brokered communication models.}}
		\item \texttt{\textbf{Existing IoT solutions are expensive.}}
		\item \texttt{\textbf{Lack of security has made users loose trust on the data sharing system.}}
		\item \texttt{\textbf{No relaible way to ensure security of collected data.}}
				\item \texttt{\textbf{Cloud
servers will remain a bottleneck and point of
failure that can disrupt the entire network.}}

	\end{itemize}
\end{frame}

\section{Solution using decentralization}

{
    \metroset{titleformat frame=smallcaps}
\begin{frame}{Decentralizing IoT networks}

      A decentralized approach to IoT networking
would solve many of the issues above.

		\begin{itemize}
		\item \texttt{\textbf{prevent failure in any single node in a network from bringing the entire network to a
halting collapse.}}
		\item \texttt{\textbf{reduce the costs
associated with installing and maintaining
large centralized data centers.}}
		\item \texttt{\textbf{IoT security is much
more than just about protecting sensitive data.}}

\item \texttt{\textbf{Any decentralized approach must support three
foundational functions:
\begin{enumerate}
\item Peer-to-peer messaging.
\item Distributed file sharing.
\item Autonomous device coordination.
\end{enumerate}
}}

	\end{itemize}
\end{frame}
}

{

\begin{frame}{The Blockchain Approach}
	\texttt{\textbf{\textit{Blockchain distributed ledger technology.}}}

The data recorded are transparent, secure, auditable, and efficient.
\metroset{block=fill}
      \begin{alertblock}{What do blockchain means?}
        \begin{itemize}
        \item \texttt{distributed ledger}
        \item \texttt{maintaining a permanent and tamper-proof record of transactional data.}
        \item \texttt{Each of the computers in the distributed network maintains a copy of the ledger}
        \end{itemize}
      \end{alertblock}

\end{frame}
}


{
\begin{frame}{Some advantages of blockchain?}
\begin{itemize}

\item The big advantage of blockchain is that it's
public.
\item A blockchain is decentralized, so there is no
single authority
\item Most importantly, it's secure. The database
can only be extended and previous records
cannot be changed

\end{itemize}
\end{frame}
}

\section{How does it work?}

\begin{frame}[fragile]{Figure 2}
  \begin{figure}  
	\includegraphics[width=11cm]{main-chain}    
    \caption{Blockchain basic image}  %\end{figure:2}
\end{figure}
\end{frame}

\begin{frame}{Block structure}

  \begin{itemize}
    \item Block ID
    \item Timestamp
    \item Nonce
    \item Data
    \item Previous block hash
   \end{itemize}
     \begin{figure}  
	\includegraphics[width=7cm]{blockchain-diagrams-02}    
    \caption{Block structure}  %\end{figure:2}
\end{figure}
\end{frame}

%\begin{frame}{Lists}
 % \begin{columns}[T,onlytextwidth]
  %  \column{0.33\textwidth}
 %     Items
   %   \begin{itemize}
    %    \item Milk \item Eggs \item Potatoes
    %  \end{itemize}
%  \end{columns}
%\end{frame}

\begin{frame}{Modification of Data}

     \begin{figure}  
	\includegraphics[width=10cm]{block_change_hash}    
    \caption{When Mutation of data happens.} 
\end{figure}
\end{frame}

%\begin{frame}{Animation}
 % \begin{itemize}[<+- | alert@+>]
  %  \item \alert<4>{This is\only<4>{ really} important}
   % \item Now this
    %\item And now this
 % \end{itemize}
%\end{frame}

\section{How blockchain can be used to secure IoT data.}

\begin{frame}{Trusted Requirements}
	IoT data as a spacial commodity, collected by government, corporates, even individuals, which are of greate value to different application fields. Such owner need a trusted platform to exchange there IoT data.

  \begin{itemize}[<+- | alert@+>]
        \metroset{block=fill}
     \item[] \begin{exampleblock}{Trusted Trading}
        \begin{itemize}
		  \item Transaction once conformed can not be modified.
		  \item Should not be maintained by a third-party.
		  \item Exchange data should be transparent.
 		\end{itemize}
      \end{exampleblock}
         \item[] \begin{exampleblock}{Trusted Data Access}
			 Data owner can hold their ownership.
      \end{exampleblock}
        \item[] \begin{exampleblock}{Trusted Privacy Preserve.}
        Data owner can protect their personal information while data exchange.
      \end{exampleblock}
  \end{itemize}
\end{frame}

\begin{frame}{Architecture}
  The framework can be divided into Data Layer, Network Layer, Protocol Layer and Interaction Layer.
	\begin{figure}  
	\includegraphics[width=7cm]{architecture}    
    \caption{Architecture of blockchain based IoT data exchange platform}
    \end{figure}
\end{frame}
\begin{frame}{Layers}
	\begin{alertblock}{Data Layer}
		 Consists of multiple network and blockchain network:
			\begin{itemize}
			\item Multiple network is responsible for origin data access and transmission.
			\item Blockchain network composed of one or more blockchain node.
			\end{itemize}
	\end{alertblock}
		\begin{alertblock}{Network Layer}
		 Consists of two parts:
			\begin{itemize}
			\item IoT data : Stored in any place the user wants.
			\item Exchange data : Stored in blockchain.
			\end{itemize}
	\end{alertblock}


\end{frame}

\begin{frame}{Layers}
	\begin{alertblock}{Management Layer}
		 
			\begin{itemize}
			\item Data Management
			\item User Management
			\item Exchange Management
			\end{itemize}
	\end{alertblock}
	\begin{alertblock}{Interaction Layer}
		 Provides the interface for data exchange parties to communicate with each other.
	\end{alertblock}	 
\end{frame}
\section{Component Design}

\begin{frame}{Exchange Management Contracts}
	Exchange management contracts include three type protocols:
			\begin{itemize}
			\item Access Contract: Uses capability based access control method to provide a trusted data permission management. 
			\item Communication Contract: Record the whole communicated process in IoT data exchange for traceability.
			\item Auto Exchange Contract: Send the data access right to demander while they satisfy the condition.
			\end{itemize}
\end{frame}

\begin{frame}{Data {\&} User Management Contracts}
	\begin{figure}  
	\includegraphics[width=5cm]{smatcontract_management}    
    \caption{Architecture of smart contract based management component}
    \end{figure}
     \begin{alertblock}{Data Management Contracts}
    		\begin{itemize}
			\item Data Contract: Generate a \emph{data object contract} and \emph{call data access contract}.  
			\item Classified search Contract: Record the whole communicated process in IoT data exchange for traceability.
			\end{itemize}
		\end{alertblock}
	\begin{alertblock}{User Management Contracts}
	Controls the users's security and permissions of the platform. 
	\end{alertblock}		
\end{frame}

\section{Conclusion}

\begin{frame}{Summary}
IoT data exchange was divided  into three categories: 
\begin{itemize}
\item trusted trading
\item trusted data access 
\item trusted privacy preserve.
\end{itemize}
\hspace{5mm}It proposes a blockchain based solution
to meet such requirement and designs the major trusted
components.\\
\hspace{5mm}Blockchain network can record the transaction in an auditable,transparent and immutable way.
  \begin{center}\ccbysa \end{center}
\end{frame}

\begin{frame}[standout]
  Questions?
\end{frame}

\appendix


\begin{frame}[allowframebreaks]{References}

	\begin{enumerate}
		\item A Decentralized Solution for IoT Data Trusted Exchange Based-on Blockchain by Zhiqing Huang , Xiongye Su at 2017 3rd IEEE International Conference on Computer and Communications.
		\item \href{https://medium.com/s/story/how-does-the-blockchain-work-98c8cd01d2ae}{How Does the Blockchain Work?}
		\item \href{https://blockgeeks.com/guides/what-is-blockchain-technology/}{What is Blockchain Technology? A Step-by-Step Guide For Beginners}
		\item \href{https://www.bbvaopenmind.com/en/securing-the-internet-of-things-iot-with-blockchain/}{Securing the Internet of Things (IoT) with Blockchain}
	\end{enumerate}

\end{frame}

\end{document}
