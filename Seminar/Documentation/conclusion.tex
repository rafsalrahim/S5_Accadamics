
\chapter{Conclusion and future scope}

Blockchain and IoT convergence will become a necessity at some point. If the current IoT paradigm — devices connected via a centralized cloud storage and processing service — continues, then systems are likely to become increasingly bloated, as data volumes, as well as the number of connected devices, continue to increase.

\paragraph{}These cloud services are likely to become bottlenecks as the amount of data pumped through them increases. Blockchains can remedy this thanks to their distributed nature.

\paragraph{}Adopting a standardized peer-to-peer communication model to process the hundreds of billions of transactions between devices will significantly reduce the costs associated with installing and maintaining large centralized data centers and will distribute computation and storage needs across the billions of devices that form IoT networks. This will prevent failure in any single node in a network from bringing the entire network to a halting collapse.

\paragraph{}Hence, the merger of these two trending technologies is inevitable. These two technologies complement as well as support each other. It is even being said that IoT needs Blockchain and Blockchain needs IoT.


